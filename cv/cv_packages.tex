%% Formatting packages-----------%-------------------------------
\providecommand{\printname}{Jonathan C.\ Lighthall}
\newcommand{\shortname}{J.\ C.\ Lighthall}

%% Load and modify settings from the letter template
%% Set indentation here
%% The letter should either be indented (each paragraph, address, and signature) or not at all
\def\indent{3in}
%\setlength{\parindent}{0in}
\InputIfFileExists{../ltr/letter_layout}{\graphicspath{{../ltr/}}}{%
\InputIfFileExists{../../ltr/letter_layout}{\graphicspath{{../../ltr/}}}{}}
 %path to signatures

%\addtolength{\textwidth}{0.5in}	%increase text width to 9in
%\addtolength{\hoffset}{-0.5in} %decrease left margin to 1.5in

\newcommand{\boldtext}[1]{\textbf{\scshape\normalsize#1}}
\newcommand{\desctext}[1]{\textit{\normalsize#1}}

\renewcommand{\nameaddress}[1]{
\boldtext{\printname}\\
%\addressprefix
#1%
%\homeaddress
%\workaddress
\addresssuffix
}

\renewcommand{\authname}{\printname}
\renewcommand{\doctitle}{\shortname~CV}
\hypersetup{
  pdfauthor=\authname,           %Adds author name to PDF properties
  pdftitle=\doctitle,            %Adds thesis title to PDF properties
  pdfstartpage=1,                %Opening page number (absolute)
  pdfstartview={XYZ null null 1},%view page at 100%
	urlcolor=blue,                  
}

%% Bibliography packages---------%-------------------------------
\usepackage{multibib}
%% Multibib generates multiple bibliographic sections.  Each bibliography requires a separate
%% .bbl file which is generated from an .aux file by BibTex.  This can be done manually or with
%% a script.  For example, create a batch file in the current directory with the following
%% contents:
%% for %%f in (*.aux) do "C:\Program Files\MiKTeX 2.9\miktex\bin\bibtex.exe" %%~nf
%% Then, define a new output profile (in TeXnicCenter) which points to the new batch file.
%\IfFileExists{multibib.bat}{}{\errmessage{Batch file for multibib not found}}
\newcites{talk}{Presentations}
\newcites{proc}{Proceedings}
\newcites{invi}{Invited Talks}

%% Application/CV settings-------%-------------------------------
\usepackage{lastpage}
\usepackage{sectsty}
\sectionfont{\boldtext} %use 

\newcommand{\heliosonly}[1]{#1}
\newcommand{\jclonly}[1]{#1}

\newcommand{\desc}[1]{%this environment provides an added description of the current item
\renewenvironment{quote}{%
\list{}{%
    \setlength{\leftmargin}{22pt}
    %\rightmargin0pt%\leftmargin
		\setlength{\itemsep}{0pt}
	\setlength{\parskip}{0pt}
	\setlength{\parsep}{0pt}
	\setlength{\topsep}{0pt}
	\setlength{\partopsep}{0pt}
  }
  \item\relax
}
{\endlist}
\begin{quote}
%\vspace*{-0.35\baselineskip}%
%\textit
\desctext
{#1}% \the\leftmargin, \the\rightmargin
%\vspace*{-0.35\baselineskip}%
\end{quote}
}

\renewenvironment{itemize}%
{ \begin{list}%
   {\textbullet}% set bullet type here
   {%
	%\setlength{\topsep}{0pt}
	%\setlength{\partopsep}{0pt}
  \setlength{\itemsep}{0pt}
  \setlength{\parskip}{0pt}
  \setlength{\parsep}{0pt}
		}
		}%
{ \end{list} }

\newcommand{\extra}[1]{{\color{note_gray}#1}} %print extra items in gray
%\renewcommand{\extra}[1]{#1} %Enters "notes" into text
\renewcommand{\extra}[1]{} %Deletes "notes"
