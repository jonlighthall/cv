\clearpage
\pagestyle{fancy}
\lhead{\boldtext{\shortname}}
\rhead{\headtext{Curriculum Vit\ae}}

\begin{changemargin}{\indent}
\nameaddress{\workaddress}%\\
\end{changemargin}
%% Turn added text on and off
%\renewcommand{\desc}[1]{}%turn off
\renewcommand{\desctext}[1]{#1}%no formatting
%\renewcommand{\desctext}[1]{\textrm{\small#1}}%small font
\renewcommand{\extra}[1]{}

%%-----------------------------------------------------------------------------------------------
\extra{%
\section*{Personal Information}
\addcontentsline{toc}{section}{Personal Information}
\begin{quote}
  \setlength{\itemsep}{0pt}
  \setlength{\parskip}{0pt}
  \setlength{\parsep}{0pt}
	Born October 7$^{\textrm{th}}$, 1981 in Ann Arbor, Michigan.  U.S. citizen.  Married to Holly Lighthall.
\end{quote}
}
%%-----------------------------------------------------------------------------------------------
\extra{Not for publication. Superfluous items printed in gray.}
\objective{%I am looking to use my skills and  %collaboration skills and 
 experience %working with diverse collaborations from around the wor
in programming,  analysis, and collaboration %tical skill 
to contribute to acoustic simulations at the Naval Research Laboratory.
 My  15 years of experience  as an experimental 
 nuclear physicist 
have required me to acquire a wide variety of %many diverse 
skills.
Specifically, these skills include 
%as an experimental physicist
%specializing in the 
the writing and development of computer code for the acquisition, calibration, simulation, and analysis of nuclear %experimental 
data;
 % from nuclear physics experiments. 
comparing simulations to experimental data;
comparing data to physical models; 
and working with diverse collaborations to report and evaluate the analysis and results of experiments.
I am interested in leveraging my education, experience, and skills to help further the practical research and development at NRL
and transition the results into operational use.
}
\phantomsection
\addcontentsline{toc}{chapter}{Curriculum Vit\ae}
%\vspace*{-1\baselineskip}
\section*{Education}
\addcontentsline{toc}{section}{Education}
%\begin{tabular}{r@{ }l@{, }l@{, }l}
%Ph.D. &Experimental Nuclear Physics&Western Michigan University& June 2011\\
%B.A.& Physics, Applied Mathematics& Albion College& May 2004\\
%&&Seaholm High School & June 2000 
%\end{tabular}
\begin{itemize}
  \item Ph.D. in Experimental Nuclear Physics, Western Michigan University, June 2011
      \extra{\\Graduation ceremony on June 25; started August 2004 (7 years, 6.9 calendar years) Age 22--29}
	  \desc{Dissertation: \href{http://gradworks.umi.com/34/71/3471048.html}{Commissioning of the Helical Orbit Spectrometer%: A new device for measuring nuclear reactions in inverse kinematics
	}}
	\item B.A. in Physics and Applied Mathematics, Albion College, May 2004 \extra{\\ Graduated May 8, started August 2000 (4 school years, 4+7=11 years total) Age 18--22\\3.21 GPA overall}%
	\extra{\item High School Diploma, Seaholm High School, June 2000} \extra{\\1996--2000, 9$^\textrm{th}$--12$^\textrm{th}$ grade, (4 school years, 4+4+7=15 years total) Age 14--18}
	\extra{\item Derby Middle School 1993--1996} \extra{\\6$^\textrm{th}$--8$^\textrm{th}$ grade (3 school years, 3+4+4+7=18) Age 11--14}
	\extra{\item Harlan Elementary School 1987--1993} \extra{\\Kindergarten and 1$^\textrm{st}$--5$^\textrm{th}$ grade (6 school years, 6+3+4+4+7=24) Age 5--11}
	%92-93 5th
	%91-92 4
	%90-91 3
	%89-90 2
	%88-89 1st grade
	%87-88
	\extra{\item Pembroke Elementary School 1986--1987} \extra{\\Developmental kindergarten (1+6+3+4+4+7=25) Age 4--5}
	\extra{\item St. James Preschool 1984--1986} \extra{\\(2+1+6+3+4+4+7=27) Age 2--4}
\end{itemize}
%%-----------------------------------------------------------------------------------------------
\section*{Research Experience}
\addcontentsline{toc}{section}{Research Experience}
\begin{itemize}
	\item Research Physicist, U.S. Naval Research Laboratory, Acoustics Simulations, Measurements and Tactics Branch, October 2018---Present
	    \desc{}
	\item Senior Postdoctoral Researcher, Department of Physics \& Astronomy, Louisiana State University\\ (C.\ Deibel), May 2016---October 2018
		\desc{Visiting Researcher at Florida State University.
		Principle investigator for the $^{17}$F($\alpha$,$p$) measurement using  the ANASEN detector. 
		Established and maintained a code repository for the calibration, interpretation, and analysis of ANASEN data (C++, ROOT, Git).
		Evaluated and developed calibration and analysis algorithms for ANASEN.
		Operated the accelerator and data acquisition for several experiments at FSU.
		Installed and commissioned the Yale split-pole spectrograph (SPS). Simulated reactions with the SPS for experimental proposals (FORTRAN, ROOT).}
	\item Postdoctoral Research Fellow, Science Division, TRIUMF (B.\ Davids), October 2012--April 2016 \extra{\\Started October 27, 2012}
		\desc{Commissioning of the EMMA spectrometer:
		%wrote 
		principle investigator on experimental proposals for RIB commissioning;
		wrote and developed code for the acquisition, calibration, simulation, and analysis of data from the EMMA gas-filled detectors (MIDAS, ROOT, SVN);
		wrote the safety report for the EMMA gas-handling system;
		developed simulations (Geant4, ANSYS);
		polished  and assembled electrostatic components;
		%data analysis for $^{13}$C experiment.
		installed radiation shielding; % September 2013--January 2014,
		designed the detector readout electronics (NIM, VME). %January 2013
		%Supervised student projects.
		Operated data acquisition and provided online data analysis for several IRIS experiments and one TUDA experiment.
		}
	\item Postdoctoral Research Associate, Physics Department, WMU (A.\ Wuosmaa), July 2011--October 2012 \extra{\\Started July 6, 2011; \$38,502 annually}
	  \desc{%Stationed at Argonne.
		Characterized silicon detectors for upgraded HELIOS array.  Developed hardware for HELIOS detector upgrades. 
		%Actively participated in
		Installed detectors and electronics and operated data acquisition for the 
		HELIOS, the Enge split-pole spectrograph, and Gammasphere
		experimental collaborations to study nuclear structure and nuclear astrophysics.}
	\item Laboratory Graduate, Physics Division, Argonne National Laboratory (B.\ Back), July 2007--June 2011 \extra{\\\$24,000 annually}
			\desc{Commissioned the HELIOS spectrometer:
		Principle Investigator on the first two HELIOS experiments;
		wrote and developed the code for the acquisition, calibration, and analysis of HELIOS data (ROOT, C++);
		designed and built the HELIOS detector readout system (NIM, CAMAC);
		wrote the HELIOS slow-control programs (LabVIEW);
		designed and developed kinematics computation programs and Monte Carlo simulations (Excel, C++);
		compared  simulations with experimental results;
		assembled and installed beam line components;
		performed detailed measurement and analysis of the solenoid field.
		Installed detectors and electronics and operated data acquisition for the 
%		Actively participated in 
				experimental collaborations using HELIOS, the Enge split-pole spectrograph, and Gammasphere.
		 }
	\pagebreak[0] \section*{\hspace{-1.25em} Research Experience (continued)}	
	\item Research Assistant, Physics Department, WMU (A.\ Wuosmaa), August 2004--June 2007
	\extra{\\Associated and supervised August 2004--June 2011 \$10,573 for 2006 \$13798.40?}
		\desc{In addition to work at Argonne---collaborated on experiments using the tandem Van de Graaff accelerator
		%with the goal of improving the pair production branching ratio of the Holye State in $^{12}$C. 
		 including NIM electronics setup, data acquisition and analysis (SpecTcl), and accelerator operation.
		Characterized APEX silicon detectors, including in-beam testing. %, data acquisition 
		%and tandem Van de Graaff operation. 
		Designed and machined detector mounts. 
		% at WMU%and ANL
				Received two research awards (see below).}
	\pagebreak[0] \section*{\hspace{-1.25em} Research Experience (continued)}	
	\item Foundation for Undergraduate Research, Scholarship, and Creative Activity (FURSCA)\\ Albion College (C.\ Moreau), May 2003--August 2003
	  \desc{Assembled and characterized the high-vacuum system for the Albion metallic thin film deposition chamber and wrote the vacuum-monitoring software (LabVIEW).}
	\item Center for Fundamental Materials Research (CFMR) Academic Affiliates Program\\Michigan State University (N.\ Birge), May 2002--August 2002
			\desc{Refurbished a 2\,kA 
		thermal evaporation system, including high-pressure water cooling, high-vacuum, and electrical systems.}
	\item Physics Department, Albion College (C.\ Moreau), January 2002--May 2004
		\desc{Assembled and characterized a thin-film deposition chamber, including high-vacuum system.}
\end{itemize}%
%\pangram{10}
%%-----------------------------------------------------------------------------------------------
%\pagebreak[0]
\section*{Teaching Experience}
\addcontentsline{toc}{section}{Teaching Experience}
\begin{itemize}
\item Visiting Researcher, FSU, May 2016---Present
		\desc{Supervised and collaborated with the team of undergraduate and graduate students on ANASEN experiments.}
\item Postdoctoral Research Fellow, TRIUMF, October 2012--April 2016
\desc{Supervised and mentored high school students, undergraduate engineering Co-op students, and physics graduate students on %various
 projects related to EMMA commissioning.}
	\item Laboratory Graduate and Postdoctoral Research Associate, Argonne, July 2007--October 2012%
	\extra{(postdoc from July 2011) worked with Lauren Perez summer 2012; worked with Jack Winkelbauer summer 2007, 2008; Adam Van Fall, Winter 2007, spring, summer 2008, summer 2009; Anna Woodard Fall 2008, Shadi}%
	  \desc{Supervised and mentored Science Undergraduate Laboratory Internship (SULI) %physics undergraduate 
	  students and physics graduate students on projects related to the commissioning and operation of HELIOS, usually working with one student per semester.}
	\extra{\item Laboratory Graduate, Argonne, July 2007--June 2011 %
	  \extra{}
	  \desc{}}
	\item Course Instructor (Physics 107), WMU, January 2006--May 2006 \extra{1070}
	\desc{Sole instructor of survey class for non-majors, class of 46 students.
	Wrote and gave lectures; assigned homework, and wrote exams.
	Received exceptional instructor evaluations and two teaching awards (see below).}
	\item Laboratory Instructor (Physics 100, 108), \extra{1000, 1080}% Physics Department,
	 WMU, August 2004--December 2005
	\desc{Ran 2--3 laboratory sections each semester for two survey classes for non-majors, 20 students per class.
	Wrote and developed supplemental instructional materials for each topic covered.
	Received exceptional instructor evaluations and one teaching award (see below).
	Worked as a tutor in the Physics Help Room and tutored local high-school and undergraduate students in physics. }
	\item Departmental Tutor (Physics, Mathematics), Albion College, January 2003--May 2004
	\desc{Worked in the Quantitative Skills Center as a tutor in math, physics, computer science, and chemistry.}
	\item Teaching Assistant (Physics 167, 168), Albion College, January 2001--May 2004
	\desc{Laboratory assistant and paper-grader for calculus-based mechanics and E\&M for physics majors.}
\end{itemize}

%%-----------------------------------------------------------------------------------------------
\section*{Awards and Honors}
\addcontentsline{toc}{section}{Awards and Honors}
\begin{itemize}
	\item 2010--2011 Leo R.\ Parpart Physics Endowed Scholarship, WMU
	\desc{Awarded annually by the Physics Department faculty for outstanding research in physics.
	%Awarded April 21, 2011; \$400 prize.		
	%\url{http://tesla.physics.wmich.edu/Degree_Programs/Graduate/Awards_Scholarships_and_Fellowships/}
	}
	
	\item 2008 All-University Research and Creative Activities Award, WMU 
	\desc {Awarded by the Office of the Vice President for Research and the Graduate Student Advisory Committee for outstanding poster presentations at the annual WMU Research and Creative Activities Poster Day.  \extra{Of over 200 participants, 13 awards were given.}
	%Awarded April 11, 2008; \$200 prize
	%\url{http://www.wmich.edu/wmu/news/2008/05/023.html}
	}
	\pagebreak[0]\section*{\hspace{-1.25em} Awards and Honors (continued)}	
	\item 2005--2006 Graduate Teaching Effectiveness Award in Physics, WMU
	\desc{Awarded annually by the Graduate College. One nomination per department.\extra{
	The Graduate Studies Council established the Graduate Teaching Effectiveness Award to recognize effective teaching that directly promotes and facilitates student learning.  Nominations are sought on an annual basis by the Graduate College. Each department may nominate one student.
	%\url{https://www.wmich.edu/grad/funding/gsc_awards.html}
	}
	}
	
	\item 2005--2006 Haym Kruglak Graduate Student Teaching Excellence Award, WMU
	\desc{Awarded annually by the Graduate Committee to physics graduate students based on the cumulative record of the student's teaching evaluations and faculty recommendations. \extra{This award is given to a physics graduate student who has demonstrated outstanding effort and accomplishments in teaching.  
	Awarded April 19, 2006; \$1,000 prize.
	\url{http://tesla.physics.wmich.edu/Degree_Programs/Graduate/Awards_Scholarships_and_Fellowships/}
	}
	}
			\pagebreak[0]\section*{\hspace{-1.25em} Awards and Honors (continued)}		
	\item 2004--2005 Jacob Dewitt Physics Teaching Award, WMU
	\desc{Awarded annually by the Graduate Committee based on evaluations by students and faculty, recognizing exemplary performance in teaching. \extra{ After the Spring semester of each academic year, the faculty members of the Graduate Committee shall meet and determine if awards for outstanding teaching have been merited. Each award recognizes exemplary performance in teaching during the Fall and/or Spring semesters of that year. All graduate students who serve as teaching assistants in physics courses at WMU during either or both of those semesters are eligible to receive the award. The decisions will be based upon student evaluation forms. Each recipient of an award will receive a congratulatory letter and a monetary award. A student may receive this award more than once.
	Awarded April 14, 2005; \$250 prize.
	\url{http://tesla.physics.wmich.edu/Degree_Programs/Graduate/Awards_Scholarships_and_Fellowships/}
	}
	}
	\item 2004 Outstanding Senior Physics Major Award, Albion College
	\desc{Awarded annually by the faculty, the E.\ T.\ S.\ Walton Award is a prize \extra{established by Nobel Laureate E.\ T.\ S.\ Walton and is} given to the outstanding senior physics major. \extra{This award consists of a technical book and is announced at the Honors Convocation in April.
	\url{http://www.albion.edu/academics/departments/physics/scholarships-and-awards}
	Awarded April 15, 2004
	}
	}
	\item 2003 Physics Alumni and Faculty Scholarship Award, Albion College
	\desc{Awarded annually by the faculty to physics or pre-engineering students who have demonstrated potential for mastery of classical and modern physics. \extra{ This scholarship is given each year to current or incoming students who are on either a physics or pre-engineering track and who have demonstrated potential for mastery of classical and modern physics. Financial need will be considered in the review of scholarship applications. The scholarship winner is announced at the Honors Convocation in April.
	\url{http://www.albion.edu/academics/departments/physics/scholarships-and-awards}
	Awarded April 24, 2003
	}
	}
	
	\item 2000--2004 E.\ R.\ Sleight Mathematics Scholarship, Albion College
	\desc{Awarded annually by the faculty to a student who demonstrates an excellence in mathematics and service to the department.
	\extra{Awarded December 7, 2000, March 21, 2002, April 2003
	\$1,200 award}
	}
	
	\extra{\item{2000--2003 Dean's List, Albion College%
	%Dean's List Fall 2000, Fall 2002, Spring 2003, Fall 2003
	%Spring 2001 warning; academic probation Fall 2001--Spring2002?
	%Social probation Fall 2003--Spring 2004
	}
	\desc{Students named to the Albion College Dean's List must achieve a grade point average of 3.5 or above at the completion of the semester. This GPA must be based on at least three units in graded courses and a minimum of four units completed.}}
	
	\extra{\item{2000 Michigan Interscholastic Press Association, Award of Merit, First Place Use of Photography, Class A}
	\desc{Spring Conference, April 10, 2000}}
	\extra{\item{Outstanding achievement in Visual Communication Technology}
	\desc{May 24, 2000}}
	\extra{\item{Michigan Competitive Scholarship, Michigan Higher Education Assistance Authority, acknowledgment of outstanding academic achievement}}
	
\end{itemize}

%%-----------------------------------------------------------------------------------------------
\section*{Professional Organizations}
\addcontentsline{toc}{section}{Professional Organizations}
\begin{itemize}
 	\item American Physical Society, joined July 2007 \extra{\photodate\formatdate{1}{07}{2007} Joined June 19, 2007}
	\item Sigma Xi,The Scientific Research Society, joined May 2004 \extra{\photodate\formatdate{1}{5}{2004}}
	%\desc{}
	\item Alpha Lambda Delta, National Academic Honor Society for Freshmen, joined November 2001 \extra{\photodate\formatdate{18}{11}{2001} November 18, 2001} %Nov 18 2001
	%\desc{}
\end{itemize}
%\begin{tabular}{lll}
%~~~~\textbullet ~~American Physical Society & & joined Jul 2007 \extra{joined June 19, 2007}\\
	%\textbullet ~~Sigma Xi & The Scientific Research Society& joined May 2004\\
	%\textbullet ~~Alpha Lambda Delta & National Academic Honor Society for Freshmen& joined Nov 2001 \extra{Nov 18 2001}
%\end{tabular}
