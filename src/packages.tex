%%-------------------Emacs PostScript "pretty-print" page width (97 columns)-------------------%%
%%3456789112345678921234567893123456789412345678|5123456789612345678971234567898123456789912345%%
%  This file loads a number of standard packages.  Also included at the end of the file are a 
%  number of macros.
%% Math packages-----------------%-------------------------------
\usepackage{amsmath}
\usepackage{bm} % bold math
%  The following provides symbols, script text in math mode (conflicts with mathdesign)
%\usepackage{amssymb,mathrsfs}

%% Text packages-----------------%-------------------------------
\usepackage{lmodern} %enhanced versions of the Computer Modern fonts
\usepackage[utf8]{inputenc} %input encoding for most languagses 
\usepackage[T1]{fontenc} %modern font encoding

%% Symbol font packages----------%-------------------------------
\usepackage{textcomp} %text symbols
%\usepackage{ulsy} %for special symbols
%\usepackage{latexsym}
%\usepackage{pifont} %symbol fonts
\usepackage{wasysym} % astonomy symbols
\usepackage{combelow} %support for Romanian letters

%% The default font is Latin Modern (replaces Computer Modern).  Following are a number of fonts 
%  with extensive math and symbol support.  To use a particular font, uncomment the associated
%  line.  If mathdesign fonts are used, the amssymb and mathrsfs packages should not be loaded
%  independently.

%% PostScript Fonts--------------%-------------------------------
%\usepackage{times} %obsolte font, use mathptmx
%\usepackage[osf]{mathpazo} %URW Palladio (A Palatino/Book Antiqua clone), old style serif 
%\usepackage{mathptmx}	%URW Nimbus Roman (A Times clone), a transitional serif typefaces
%\usepackage[scaled]{helvet}	%URW Nimbus Sans (A Helvetica clone), a sans-serif typeface 
%\renewcommand*\familydefault{\sfdefault} %% Only if the base font of the document is to be sans serif

%% Math Design Fonts-------------%-------------------------------
%\usepackage[urw-garamond]{mathdesign} %Old style serif typeface
\usepackage[adobe-utopia]{mathdesign} %Transitional serif typefaces (professional)
%\usepackage[bitstream-charter]{mathdesign} %Glyphic serif typeface optimized for low-resolution printing

\usepackage[normalem]{ulem}%for strikethroughs

%% Figure packages---------------%-------------------------------
%  The following lines of code assume that the graphics files are .eps files located in a
%  directory named Figures.  Black and white versions of color graphics files end in _bw.  For
%  example, the two files
%			/Figures/foo.eps
%			/Figures/foo_bw.eps
%  may be included with a single command, e.g., \includegraphics{foo}.  The black and white
%  figures are then selected by uncommenting the declare graphics extension rule below. In 
%  previous versions of epstopdf, it was required to specify the .eps conversion rules. Now those
%  commands destroy both the original .eps file and the .pdf output. It used to be that in order
%  for these commands to work, the tag -enable-write18 had to be included in command line 
%  arguments to make PDF figures (deprecated)

\usepackage{ifpdf}
\ifpdf
	\pdfminorversion=7 %PDF version compatability
	\usepackage{graphicx}
	\usepackage[update]{epstopdf}
	%\pdfimageresolution=10
	%\pdfcompresslevel=9 
	%\epstopdfDeclareGraphicsRule{.eps}{pdf}{.pdf}{ps2pdf -dEPSCrop #1 \OutputFile}
	%\epstopdfDeclareGraphicsRule{_bw.eps}{pdf}{_bw.pdf}{ps2pdf -dEPSCrop #1 \OutputFile}
        %\epstopdfsetup{suffix=-\SourceExt-converted-to}%This line explicitly defines the default behavior
        \epstopdfsetup{suffix=}%
\else
\usepackage{graphicx}	
\usepackage[all,light]{draftcopy}% Places "DRAFT" on each page.  Only works with PS output.	
\fi
%\DeclareGraphicsExtensions{_bw.eps,.eps}	%Turn on/off BW figures, assumes BW figures end in _bw.eps
%\graphicspath{{Figures/}}%set graphics path(s) here
\graphicspath{{Figures/}{../Figures/BW_Figures/}{../Figures/}} %Uniquely-named B&W figures

\def\figwid{4.9in}
\usepackage{wrapfig}

%% Table packages----------------%-------------------------------
\usepackage{dcolumn}% Align table columns on decimal point
\newcolumntype{d}[1]{D{.}{.}{#1}}
\newcolumntype{.}{D{.}{.}{-1}}
\newcolumntype{,}{D{,}{,}{-1}}
\newcolumntype{)}{D{)}{)}{-1}}
\newcolumntype{-}{D{-}{-}{-1}}

\renewcommand{\arraystretch}{1.3} % set tabular line spacing

\usepackage{multirow}
%\usepackage{longtable}
\usepackage{supertabular}
%\usepackage{multicol}
%\usepackage{stabular}
\usepackage{rotating}
\usepackage{pdflscape}
\usepackage{tabulary}
%\usepackage{colortbl}

%% Formatting packages-----------%-------------------------------
\usepackage{fancyhdr} %required for non-standard (fancy) headers
\usepackage{fancyvrb}%allows verbatim in footnotes
%  Linux-like command prompt "$" 
\newcommand{\vsetlinux}{\renewcommand{\FancyVerbFormatLine}[1]{{{\color{green} \$ }}##1}}
%  DOS-like command prompt
\newcommand{\vsetdos}{\renewcommand{\FancyVerbFormatLine}{{\color{blue}\texttt{C:\char`\\ >} }}}
%  ROOT-like command prompt
\newcommand{\vsetroot}{\renewcommand{\FancyVerbFormatLine}[1]{{\color{blue}root [\ifthenelse{\value{FancyVerbLine} = 0}{\emptyset}{\arabic{FancyVerbLine}}] }##1}}
%  reset Verbatim
\newcommand{\vsetnone}{\renewcommand{\FancyVerbFormatLine}[1]{##1}}
\usepackage{indentfirst}%indents first paragraph of each section
\usepackage[pagewise]{lineno} %prints line numbers with the command \linenumbers

%% Listings settings for C++ code blocks------------------------
\usepackage{listings}
\lstset{language=C++,
%basicstyle=\small,
basicstyle=\ttfamily \color{black},
commentstyle=\color{red},
keywordstyle=\color{green},
stringstyle=\color{magenta},
directivestyle={\color{cyan}},
identifierstyle={\color{blue}},
showstringspaces=false,
%emptylines=1,
numbers=left,
%numbersep=5pt,                   % how far the line-numbers are from the code
numberstyle=\small\color{note_gray}, % the style that is used for the line-numbers
rulecolor=\color{note_gray},         % if not set, the frame-color may be changed on line-breaks within not-black text (e.g. comments (green here))
breaklines=true,
frame=single,                    % adds a frame around the code
title=\lstname,                   % show the filename of files included with \lstinputlisting; also try caption instead of title
captionpos=b
}

\lstset{emph={%  
TH1F, TH2F, Float_t, Int_t, TString, TCutG, TF1%
},emphstyle={\color{green}}%
}%
\renewcommand{\lstlistingname}{Code Block}

%% Hyperref settings for PDFs and URLs---------------------------
\providecommand{\authname}{Jon Lighthall}
\providecommand{\doctitle}{LaTeX document}

\usepackage{hyperref}
\hypersetup{
pdfauthor=\authname,           %Adds author name to PDF properties
pdftitle=\doctitle,            %Adds thesis title to PDF properties
pdfstartpage=1,                %Opening page number (absolute)
%pdfstartview=FitV,            %Fits the horiz. width in the window (FitV for vertical)
pdfstartview={XYZ null null 1},%view page at 100%
pdfpagemode=UseOutlines,       %Nav. Panel: UseNone, UseThumbs, UseOutlines, FullScreen, etc.
bookmarksopen=true,            %Displays Bookmarks in the Navigation Panel
bookmarksopenlevel=\maxdimen,  %\maxdimen all levels, 0 chapters, 1 sections%
bookmarksnumbered=true,        %Numbers bookmarks with section numbers
final=true,                    %keeps hyperref features in draft mode
colorlinks=true,               %colors the links instead of using boxes
urlcolor=blue,                 %makes URL hyperlinks blue (instead of pink)
linkcolor=black,               %makes internal links black (instead of red)             
citecolor=black,               %makes citation links black (instead of green)
}

% The following commands are provided in case hyperref is not used
\providecommand{\texorpdfstring}[2]{#1} 
\providecommand{\href}[2]{#2}

%% Bibliography packages---------%-------------------------------
\usepackage[numbers,sort&compress]{natbib} %standard for APS-style citations

%% Macro definitions-------------%-------------------------------
\def\newzero{{0\mkern-9mu\texttt{/}}}%This line defines a slash-zero (similar to \emptyset)
\def\newsym{${\sim\mkern-17mu\textit{{\footnotesize/}}}$}%This line defines a slash-tilde (similar to \psi)
\def\monogram{{\fontfamily{mdugm}\selectfont J\kern-.349em {\color{black}L} \kern -.747em {\scriptsize C }}}%

%% The following lines provide definitions for printing notes in a different font
\usepackage{color}
\definecolor{note_gray}{gray}{0.4}
\newcommand{\note}[1]{{\color{note_gray}\textit{#1}}} %Emphasizes notes
%\renewcommand{\note}[1]{#1} %Enters "notes" into text
%\renewcommand{\note}[1]{} %Deletes "notes"
%  The following command creates margin notes in the "notes" font (with an arrow)
\newcommand{\marnote}[1]{\marginpar[\singlespacing\note{#1 $\rightarrow$}]% 
                                     {\singlespacing\note{$\leftarrow$ #1}}}
\renewcommand{\marnote}[1]{\marginpar[\singlespacing%
                                      \raggedright \note{#1 $\rightarrow$}]% 
                                     {\singlespacing\note{\raggedleft $\leftarrow$ #1}}}

%% Set the specific formatting of the "Mev per nucleon" units symbol
\newcommand{\AMeV}{A\,MeV}
\renewcommand{\AMeV}{AMeV}
%\renewcommand{\AMeV}{MeV/u}
%\renewcommand{\AMeV}{MeV/A}
\renewcommand{\AMeV}{MeV/$u$}
\renewcommand{\AMeV}{MeV/nucleon} %new standard

%  Load autotext, an collection of automatic text-generating macros
\InputIfFileExists{src/autotext}{}{} 
\InputIfFileExists{../src/autotext}{}{} 

%\renewcommand{\thefootnote}{\fnsymbol{footnote}} %sets footnote marks to symbols

%% Date settings-----------------%-------------------------------
\usepackage{fmtcount,datetime} %provides commands for displaying times and dates
\newdateformat{photodate}{\shortmonthname[\THEMONTH].~\THEYEAR}
\newdateformat{longusdate}{%standard long date format
\dayofweekname{\THEDAY}{\THEMONTH}{\THEYEAR}, \monthname[\THEMONTH] \ordinal{DAY}, \THEYEAR}
%  Time zone definitions, for the command \timezone.
\makeatletter
\@namedef{timezone-0400}{EDT}
\@namedef{timezone-0500}{EST}
\@namedef{timezone-0600}{CST}
\@namedef{timezone-0500}{CDT}
\@namedef{timezone-0700}{PDT}
\@namedef{timezone-0800}{PST}

\def\grabtimezone #1#2#3#4#5#6#7#8#9{\grabtimezoneB}
\def\grabtimezoneB #1#2#3#4#5#6#7{\grabtimezoneC}
\def\grabtimezoneC #1#2'#3'{%
\@ifundefined{timezone#1#2#3}
{No timezone for `#1#2#3'}
{\@nameuse{timezone#1#2#3}}%
}
\def\timezone{\expandafter\grabtimezone\pdfcreationdate}% called with \timezone.
\makeatother

\newcommand{\curl}[1]{%
\begin{center}
	\url{#1}
\end{center}
}

%% --------------------------------------------------------------
%% Layout Options----------------%-------------------------------
%% --------------------------------------------------------------
%  Set non-LaTeX-standard layout formatting here. This code was previously contained in a separate file named layout.tex
\usepackage{adjustbox}
\hyphenation{Mesy-tec}
\hyphenation{feed-through feed-throughs ejec-tile ejec-tiles cor-re-spond-ing sol-en-oid-al}

%% Spacing settings
\usepackage{setspace} %required for double spacing and used to control vertical spacing 
%% The default document line spacing is single space.  For editing, set double spacing below.
%\setstretch{2} %typewriter doublesace
%\doublespacing %standard word processor doublespace
%% Note: using word processor-style double spacing with wide margins produces documents of nearly the same length as single-spaced documents with narrow (default) margins.

%% Margin settings---------------%-------------------------------
%  Single-sided
%  For narrow margins (word processor style), set margins below.
\addtolength{\textwidth}{1.5in}		%increase text width to 6in
\addtolength{\textheight}{1.75in}	%increase text width to 9in
\addtolength{\hoffset}{-0.4375in} %decrease left margin to 1.5in
\addtolength{\voffset}{-0.875in}  %decrease top margin to 1.0in
\reversemarginpar %Moves margin notes to the wider margin

%% Double-sided------------------%-------------------------------
%  For double-sided documents, include the document class option [twoside].
%  For double-sided documents with narrow margins, set margins below.  These margin settings 
%  leave the odd-page margins unchanged while increasing the even-page left-hand margins by 
%  0.5 inches so that the margins are the same on both sides of the page.
\makeatletter
\if@twoside
 \setlength{\oddsidemargin}{62pt}     %keep 1.5in left margin on odd pages
 \addtolength{\evensidemargin}{-1in}  %decrease left margin to 1.5in on even pages
 \addtolength{\marginparwidth}{-56pt} %keep 69pt margin paragraph width
 \raggedbottom
\fi
\makeatother

%% The following command allows the margins to be reduced by the following amount. Used for indenting regions.
\newenvironment{changemargin}[1]{%
\begin{list}{}{%
\setlength{\topsep}{0pt}%
\setlength{\leftmargin}{#1}%
\setlength{\rightmargin}{#1}%
\setlength{\listparindent}{\parindent}%
\setlength{\itemindent}{\parindent}%
\setlength{\parsep}{\parskip}%
}%\centering%
\item[]}{\end{list}}
