%%-------------------Emacs PostScript "pretty-print" page width (97 columns)-------------------%%
%% This file loads a number of standard packages.  Also included at the end of the file are a number of macros.
%% Math packages-----------------%-------------------------------
\usepackage{amsmath}
%\usepackage{amssymb,mathrsfs}%provides symbols, script text in math mode (conflicts with mathdesign)
%\usepackage{bm}% bold math

%% Text packages-----------------%-------------------------------
\usepackage{textcomp,lmodern}%for special symbols 
%\usepackage[normalem]{ulem}%for strikethroughs
\usepackage[utf8]{inputenc}
\usepackage[T1]{fontenc}%modern font encoding

%% The default font is Latin Modern (replaces Computer Modern).  Following are a number of fonts 
%  with extensive math and symbol support.  To use a particular font, uncomment the associated
%  line.  If mathdesign fonts are used, the amssymb and mathrsfs packages should not be loaded
%  independently.

%% PostScript Fonts--------------%-------------------------------
%\usepackage[osf]{mathpazo}		%URW Palladio (A Palatino/Book Antiqua clone), an old style serif typeface
%\usepackage{mathptmx}				%URW Nimbus Roman (A Times clone), a transitional serif typefaces
%\usepackage[scaled]{helvet}	%URW Nimbus Sans (A Helvetica clone), a sans-serif typeface 
%\renewcommand*\familydefault{\sfdefault} %% Only if the base font of the document is to be sans serif

%% Math Design Fonts-------------%-------------------------------
%\usepackage[urw-garamond]{mathdesign}			%Old style serif typeface
\usepackage[adobe-utopia]{mathdesign}			%Transitional serif typefaces (professional)
%\usepackage[bitstream-charter]{mathdesign}	%Glyphic serif typeface optimized for low-resolution printing

%% Figure packages---------------%-------------------------------
%% The following lines of code assume that the graphics files are .eps files located in a
%  directory named Figures.  Black and white versions of color graphics files end in _bw.  For
%  example, the two files
%			/Figures/foo.eps
%			/Figures/foo_bw.eps
%  may be included with a single command, e.g., \includegraphics{fig2}.  The black and white
%  figures are then selected by uncommenting the declare graphics extension rule below.

\usepackage{ifpdf}
\ifpdf
	\usepackage[pdftex]{graphicx}
	\usepackage[update]{epstopdf}
	%May need to include -enable-write18 in command line arguments to make PDF figures (deprecated)
	\epstopdfDeclareGraphicsRule{.eps}{pdf}{.pdf}{ps2pdf -dEPSCrop #1 \OutputFile}
	\epstopdfDeclareGraphicsRule{_bw.eps}{pdf}{_bw.pdf}{ps2pdf -dEPSCrop #1 \OutputFile}
	\epstopdfsetup{suffix=}%
\else
	\usepackage{graphicx}	
	\usepackage[all,light]{draftcopy}% Places "DRAFT" on each page.  Only works with PS output.	
\fi
%\DeclareGraphicsExtensions{_bw.eps,.eps}	%Turn on/off BW figures, assumes BW figures end in _bw.eps
%\graphicspath{{Figures/}}%set graphics path(s) here
\graphicspath{{Figures/}{../Figures/BW_Figures/}{../Figures/}} %Uniquely-named B&W figures
\def\figwid{4.9in}

%% Table packages----------------%-------------------------------
\usepackage{dcolumn}% Align table columns on decimal point
\newcolumntype{d}[1]{D{.}{.}{#1}}
\newcolumntype{.}{D{.}{.}{-1}}
\newcolumntype{,}{D{,}{,}{-1}}
\newcolumntype{)}{D{)}{)}{-1}}

\renewcommand{\arraystretch}{1.3} % set tabular line spacing

\usepackage{multirow}
%\usepackage{longtable}
\usepackage{supertabular}
%\usepackage{multicol}
%\usepackage{stabular}
\usepackage{rotating}
\usepackage{pdflscape}

%% Formatting packages-----------%-------------------------------
\usepackage{setspace} %required for double spacing and used to control vertical spacing 
\usepackage{fancyhdr} %required for non-standard (fancy) headers
\usepackage{fancyvrb}%allows verbatim in footnotes
\usepackage{indentfirst}%indents first paragraph of each section

\providecommand{\authname}{Jon Lighthall}
\providecommand{\thesistitle}{LaTeX document}

\usepackage{hyperref}
\hypersetup{pdfauthor=\authname,       %Adds author name to PDF properties
        pdftitle=\thesistitle,         %Adds thesis title to PDF properties
	pdfstartpage=1,                %Opening page number (absolute)
	pdfstartview=FitV,             %Fits the horiz. width in the window (FitV for vertical)
	%pdfstartview={XYZ null null 1},%view page at 100%
	bookmarksopen=true,            %Displays Bookmarks in the Navigation Panel
	bookmarksopenlevel=0,          %\maxdimen all levels, 0 chapters, 1 sections
	bookmarksnumbered=true,        %Numbers bookmarks with section numbers
	final=true,                    %keeps hyperref features in draft mode
	colorlinks=true,               %colors the links instead of using boxes
	urlcolor=blue,                 %makes URL hyperlinks blue (instead of pink)
	linkcolor=black,               %makes internal links black (instead of red)
	citecolor=black,               %makes citation links black (instead of green)
}

%% The following two lines may be required for proper formating of the Table of Figures
%\usepackage{tocloft}
%\setlength{\cftfignumwidth}{3em}%increases spacing after figure numbers in Table of Figures

%\usepackage{makeidx}
%\makeindex
\usepackage[pagewise]{lineno} %prints line numbers with the command \linenumbers
\usepackage{color}

%% Biblography packages----------%-------------------------------
\usepackage[numbers,sort&compress]{natbib}

%% Macro definitions-------------%-------------------------------
\def\newzero{{0\mkern-9mu\texttt{/}}}%This line defines a slash-zero (similar to \emptyset)
\def\newsym{${\sim\mkern-17mu\textit{{\footnotesize/}}}$}%This line defines a slash-tilde (similar to \psi)
\def\monogram{{\fontfamily{mdugm}\selectfont J\kern-.349em {\color{black}L} \kern -.747em {\scriptsize C }}}%

%% The following lines provide definitions for printing notes in a different font
\definecolor{note_gray}{gray}{0.4}
\newcommand{\note}[1]{{\color{note_gray}\textit{#1}}} %Emphasizes notes
%\renewcommand{\note}[1]{#1} %Enters "notes" into text
%\renewcommand{\note}[1]{} %Deletes "notes"
%  The following command creates margin notes in the "notes" font (with an arrow)
\newcommand{\marnote}[1]{\marginpar[\singlespacing\note{#1 $\rightarrow$}]% 
                                   {\singlespacing\note{$\leftarrow$ #1}}}

%% Set the specific formatting of the "Mev per nucleon" units symbol
\newcommand{\AMeV}{A\,MeV}
%\renewcommand{\AMeV}{MeV/u}
%\renewcommand{\AMeV}{MeV/A}
\renewcommand{\AMeV}{MeV/$u$}

\InputIfFileExists{More_Files/autotext}{}{}

\renewcommand{\thefootnote}{\fnsymbol{footnote}} %sets footnote marks to symbols
\newdateformat{photodate}{\shortmonthname[\THEMONTH].~\THEYEAR}

\providecommand{\texorpdfstring}[2]{#1} %provided in case hyperref is not used

\providecommand{\Chapter}[1]{\chapter[\texorpdfstring{\MakeUppercase{#1}}{#1}]{#1}}
