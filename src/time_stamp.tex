% This file prints the time of compiling.  Requires package datetime
\newdateformat{longusdate}{%
\dayofweekname{\THEDAY}{\THEMONTH}{\THEYEAR}, \monthname[\THEMONTH] \ordinal{DAY}, \THEYEAR}
\makeatletter

\@namedef{timezone-0400}{EDT}
\@namedef{timezone-0500}{EST}
\@namedef{timezone-0600}{CST}
\@namedef{timezone-0500}{CDT}
\@namedef{timezone-0700}{PDT}
\@namedef{timezone-0800}{PST}

\def\grabtimezone #1#2#3#4#5#6#7#8#9{\grabtimezoneB}
\def\grabtimezoneB #1#2#3#4#5#6#7{\grabtimezoneC}
\def\grabtimezoneC #1#2'#3'{%
  \@ifundefined{timezone#1#2#3}
    {No timezone for `#1#2#3'}
    {\@nameuse{timezone#1#2#3}}%
}
\def\timezone{\expandafter\grabtimezone\pdfcreationdate}
\makeatother

%\newenvironment{changemargin}[1]{%
  %\begin{list}{}{%
    %\setlength{\topsep}{0pt}%
    %\setlength{\leftmargin}{#1}%
    %\setlength{\rightmargin}{#1}%
    %\setlength{\listparindent}{\parindent}%
    %\setlength{\itemindent}{\parindent}%
    %\setlength{\parsep}{\parskip}%
  %}\centering\item[]}{\end{list}}

\begin{changemargin}{-\marginparwidth}
{%\footnotesize
At the time of compiling, today is \longusdate\today{} and the current time is \currenttime{} 
\ifthenelse{\THEHOUR<12}{{}(\amstring) }{%
\ifthenelse{\THEHOUR=12}{{}(\pmstring) }{}}% 
 \timezone.
\\\enlargethispage{\baselineskip}{\color{white}Document designed by Jonathan C. Lighthall}
}
\end{changemargin}