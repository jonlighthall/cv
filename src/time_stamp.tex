%% Date settings-----------------%-------------------------------
%  This file provides definitions and commands to prints the time of
%  compiling.

% dependencies
\usepackage{datetime} % provides commands for displaying times and dates
\usepackage{setspace} % needed to change margins

% Time zone definitions, for the command \timezone.
% On Overleaf, the timezone command requires a file named latexmkrc with
% the following contents
% #$ENV{'TZ'}='America/Detroit';
% $ENV{'TZ'}='America/Chicago';
% #$ENV{'TZ'}='America/Denver';
% #$ENV{'TZ'}='America/Vancouver';

\makeatletter
\newif\ifdst
% set daylight saving time
\DSTFALSE
% set time zone names
\ifdst
  \@namedef{timezone-0400}{EDT}
  \@namedef{timezone-0500}{CDT}
  \@namedef{timezone-0600}{MDT}
  \@namedef{timezone-0700}{PDT}
\else
  \@namedef{timezone-0500}{EST}
  \@namedef{timezone-0600}{CST}
  \@namedef{timezone-0700}{MST}
  \@namedef{timezone-0800}{PST}
\fi
\def\grabtimezone #1#2#3#4#5#6#7#8#9{\grabtimezoneB}
\def\grabtimezoneB #1#2#3#4#5#6#7{\grabtimezoneC}
\def\grabtimezoneC #1#2'#3'{%
  \@ifundefined{timezone#1#2#3}
    {No timezone for `#1#2#3'}
    {\@nameuse{timezone#1#2#3}}%
}
\def\timezone{\expandafter\grabtimezone\pdfcreationdate}% called with \timezone.
\makeatother

% Date formats
\newdateformat{photodate}{\shortmonthname[\THEMONTH]~\THEYEAR}
\newdateformat{longusdate}{%standard long date format
\dayofweekname{\THEDAY}{\THEMONTH}{\THEYEAR}, \monthname[\THEMONTH] \ordinal{DAY}, \THEYEAR}
\newdateformat{shortusdate}{%standard short date format
\shortdayofweekname{\THEDAY}{\THEMONTH}{\THEYEAR}, \shortmonthname[\THEMONTH] \ordinal{DAY}, \THEYEAR}
\newdateformat{reallyshortusdate}{%standard short date format
\THEYEAR-\THEMONTH-\THEDAY}

% Time stamps
\newcommand{\smalltimestamp}{Updated \shortusdate\today{} \currenttime{} \timezone.}
\newcommand{\reallysmalltimestamp}{\reallyshortusdate\today{} \currenttime{} \timezone}
\newcommand{\timestamp}{%
At the time of compiling, today is \longusdate\today{} and the current time is \currenttime{} 
\ifthenelse{\THEHOUR<12}{{}(\amstring) }{%
\ifthenelse{\THEHOUR=12}{{}(\pmstring) }{}}% 
\timezone.%
}
% The following command may be used on its own or use within \AtEndDocument{}
\newcommand{\terminaltimestamp}{%
%\AtEndDocument{
%% The need to include one of the following two lines may be system dependent.
%\pagebreak % use if float(s) on last page 
%\nopagebreak[4] % use if text on last page (if needed)
 \makeatletter 
  \if@twocolumn
    \tiny
      \fi
\makeatother

%% The following command allows the margins to be reduced by the following amount. Used for indenting regions.
\newenvironment{changemargintt}[1]{%
\begin{list}{}{%
\setlength{\leftmargin}{##1}%
\setlength{\rightmargin}{##1}%
}
\item[]}{\end{list}}
\begin{singlespace}
    \begin{center}
\vspace{-2\baselineskip}
        \rule[0 pt]{\linewidth}{0.5pt}\\	
        \textit{(end of document)}\\
        \vspace*{\fill}
%\vspace*{\stretch{1}}
%\begin{figure*}[h]%
%\centering
%\includegraphics[keepaspectratio,height=0.25\textheight,width=0.25\columnwidth]{monogram}%
%\end{figure*}
%\input{colophon}
%\vspace*{\stretch{1}}
        \begin{changemargintt}{-\marginparwidth}%
            {%\footnotesize
                \begin{center}
                    \timestamp
                    %%\\\enlargethispage{\baselineskip}{\color{white}Document designed by Jonathan C. Lighthall}
                \end{center}
            }
        \end{changemargintt}
    \end{center}
\end{singlespace}
%\marginpar{\color{white}\texttt{J/\textbackslash C\textbar\textless @L}}
%}
}

\newcommand{\terminaltimestamps}{
\singlespace
\centering
\setlength{\parindent}{0pt}
\rule[0 pt]{\linewidth}{0.125pt}
\textit{(end of document)}
\vspace*{\fill}
\\\timestamp
}