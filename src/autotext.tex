%% The following command is intended to produce dummy body text for testing fonts,
%  layouts, and formatting.
\usepackage{lipsum}
\usepackage{ifthen}
\newcounter{i} %jack and pangram iteration counter
\newcounter{j} %pangram iterations per paragraph counter
\newcounter{k} %autotext iteration counter
\newcounter{l} %autotext level counter
\newcounter{m}
\newcounter{n}

%% Simple text output in one, unbroken paragraph.
\newcommand{\jack}[1]{%10 words
	\setcounter{i}{0}	
	\whiledo{\value{i} < #1}{
	All work and no play makes Jack a dull boy.%
	\addtocounter{i}{1}
	%(\the\value{i})  %
	}
}

%% Pangram output.  A new paragraph is started after every tenth iteration.
\newcommand{\pangram}[1]{%9 words
	\setcounter{i}{0}%
	\setcounter{j}{0}%
	\whiledo{\value{i} < #1}{%
		The quick brown fox jumps over a lazy dog%
		\addtocounter{i}{1}%
		\addtocounter{j}{1}%
%~(\the\value{i})% print iteration number after each sentence
. % 
		\ifthenelse{\value{j} = 10}{%iterations per par
			\setcounter{j}{0}%
			\par%\marnote{This is a long, two-line margin note}
			}{}%
	 }
}
\providecommand{\Chapter}[1]{\chapter{#1}}

\newcommand{\autotext}[3]{%
	\setcounter{n}{0}%highest level -1
	\setcounter{k}{0}%iteration counter
	\setcounter{l}{\value{n}}%level counter
	\setcounter{m}{1}%chapter counter
	\whiledo{\value{k} < #1} {%
		\addtocounter{k}{1}%
	\addtocounter{l}{1}%
%		k=\the\value{k}, l=\the\value{l}, m=\the\value{m}
		\ifthenelse{\value{l} = 0}{\part{Level \the\value{l} Auto Text \the\value{m}} Text after a part and before a chapter. }{}%
		\ifthenelse{\value{l} = 1}{\Chapter{Level \the\value{l} Auto Text \the\value{m}} Chapter text\footnote{This is a footnote}%
		.  This is a citation Ref.~\cite{Lighthall_2010}. }{}
		\ifthenelse{\value{l} = 2}{\section{Level \the\value{l} Auto Text \the\value{m}} Section text. }{}%
		\ifthenelse{\value{l} = 3}{\subsection{Level \the\value{l} Auto Text \the\value{m}} Subsection text. }{}%
		\ifthenelse{\value{l} = 4}{\subsubsection{Level \the\value{l} Auto Text \the\value{m}} Sub-subsection text. }{}%
		\ifthenelse{\value{l} = 5}{\paragraph{Level \the\value{l} Auto Text \the\value{m}} Paragraph text. }{}%
		\ifthenelse{\value{l} = 6}{\subparagraph{Level \the\value{l} Auto Text \the\value{m}} Sub-paragraph text. }{}%
		\pangram{#2}%
		\begin{equation}
			x+y=\text{\the\value{k}}
		\end{equation}
		\begin{figure}%
			\centering%
\setlength{\unitlength}{0.125\columnwidth}
\fbox{
\begin{picture}(4.5,2.75)(.25,.125)
\thinlines
\put(0.5,0.50){\line(1,0){3.75}}%z
\put(0.5,0.50){\line(0,1){2}}   %rho
\thicklines
\put(0.5,0.5){\vector(1,2){1}}    
\put(0.5,0.5){\vector(3,2){2.25}}
\put(0.5,0.5){\vector(4,1){3.5}} 
\put(4.4,0.45){$z$}
\put(.45,2.7){$\rho$}
\put(3.5,2){\the\value{k}}
\put(3.55,2.05){\circle{0.5}}
\end{picture}
%text~\the\value{k}
}
			\caption{This is figure \the\value{k}.}%
		\end{figure}%
		\begin{table}%
			\centering%
			\begin{tabular}{c|c}%
				\the\value{k}&\the\value{k}\\\hline%
				\the\value{k}&\the\value{k}%
			\end{tabular}%
			\caption{This is table \the\value{k}.}%
		\end{table}%
\pangram{#2}%
				\ifthenelse{\value{l} = #3}{%The following lines balance each subdivision with a matching pair.
		%\setcounter{l}{\value{n}} 
		\addtocounter{n}{1}
	\par{}%
		\whiledo{\value{l} > \value{n}}{%
		\ifthenelse{\value{l} = 0}{\part{Level \the\value{l} Auto Text \the\value{m}B} Text after a part and before a chapter. }{}%
		\ifthenelse{\value{l} = 1}{\Chapter{Level \the\value{l} Auto Text \the\value{m}B} Chapter text. }{}%
		\ifthenelse{\value{l} = 2}{\section{Level \the\value{l} Auto Text \the\value{m}B} Section text. }{}%
		\ifthenelse{\value{l} = 3}{\subsection{Level \the\value{l} Auto Text \the\value{m}B} Subsection text. }{}%
		\ifthenelse{\value{l} = 4}{\subsubsection{Level \the\value{l} Auto Text \the\value{m}B} Sub-subsection text. }{}%
		\ifthenelse{\value{l} = 5}{\paragraph{Level \the\value{l} Auto Text \the\value{m}B} Paragraph text. }{}%
		\ifthenelse{\value{l} = 6}{\subparagraph{Level \the\value{l} Auto Text \the\value{m}B} Sub-paragraph text. }{}%
	\addtocounter{l}{-1}%
\pangram{#2}
	}
	\addtocounter{m}{1}%
	\par{}%		
	\addtocounter{l}{-1}%
	\addtocounter{n}{-1}%	
	}{}%
	}%
%\vspace*{\stretch{1}}
\begin{figure}[b!]%
\noindent\rule[0 pt]{0.4\linewidth}{0.5pt}\par
{\footnotesize{The preceding text has been generated using the command \texttt{\textbackslash autotext}.  The command takes three arguments, which have been entered as \texttt{\{#1\}\{#2\}\{#3\}}.
The first argument (\texttt{#1}) is the total number of iterations. Each iteration produces a  level heading and three floats: an equation (requires the package \texttt{amsmath}); a figure and a table.  The second argument (\texttt{#2}) is the number of test sentences printed before and after the group of floats in each level.  The third argument (\texttt{#3}) is the level number to which each chapter is expanded.  Each expansion level appears twice.
}}%
\end{figure}
}
